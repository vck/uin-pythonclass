\section{MODUL PEMROGRAMAN PYTHON}\label{modul-pemrograman-python}

\begin{center}\rule{0.5\linewidth}{\linethickness}\end{center}

oleh: Vicky Vernando Dasta

vicky.vernando@student.unri.ac.id

\section{}\label{section}

\section{}\label{section-1}

\section{}\label{section-2}

\section{Pengenalan}\label{pengenalan}

Pemrograman tingkat tinggi menekankan pada efisiensi waktu. Python
awalnya hadir dalam bentuk ide bagaimana membuat program layaknya bash
pada \textbf{nix} platform, seiring berkembangnya Python, penggunaan
Python tidak hanya pada bahasa scripting pengganti bash, namun sebagai
bahasa pemrograman tingkat tinggi yang mampu melakukan hampir semua
kebutuhan komputasi seperti simulasi saintifik, pemrosesan data dan
kebutuhan komputasi lain. Pada modul ini akan dijelaskan mengenai
penggunaan Python dari dasar sampai pembuatan aplikasi yang bisa
langsung dipakai. Python menggunakan dua konsep dalam bahasa
pemrograman, interpreted language dan compiled language. Untuk
menjalankan script \texttt{*.py} diperlukan sebuah interpreter yang bisa
diunduh di \url{http://www.python.org}.

\section{Penggunaan}\label{penggunaan}

Untuk pengajaran Python, akan digunakan Google Colab, platform ini
dipilih karena tidak diperlukan instalasi lokal sehingga lebih efisien
dalam segi waktu. Sedangkan untuk penggunaan python secara lokal
(offline), bisa mengikuti tutorial instalasi Python di
\url{http://www.python.org}. Pada kelas ini, kita akan menggunakan
Python versi 3.x (\textgreater{}3.4) lebih baik, karena terdapat
beberapa perbaikan dan pembaruan fitur yang lebih baik untuk performa
program yang akan kita buat.

\section{Google Colab}\label{google-colab}

Google Colab adalah platform yang ditujukan pada researcher/peminat
machine learning, platform ini berupa Jupyter Notebook yang sudah
dibekali \emph{instance Graphic Processing Unit} (GPU) yang diberikan
oleh Google .Inc secara cuma-cuma.

Berikut cara penggunaan Google Colab:

\begin{enumerate}
\def\labelenumi{\alph{enumi}.}
\item
  Masuk ke akun Google anda
\item
  Kemudian buka \url{http://colab.research.google.com}
\item
\end{enumerate}

\section{Pengenalan bahasa pemrograman
Python}\label{pengenalan-bahasa-pemrograman-python}

\section{Python Keywords}\label{python-keywords}

\begin{quote}
print, if, else, elif, and, or, while, for, def, lambda, import, from,
as, continue, break, try, except, with, list, dict, zip, set, class, in
\end{quote}

Keyword diatas tidak dapat dijadikan sebagai nama untuk variabel di
Python, karena keyword di atas sudah \emph{reserved} sebagai
keyword/grammar Python. Untuk penamaan variabel yang memiliki nama yang
sama dengan reserved keyword, disarankan menggunakan \emph{underscore}
\texttt{\_\_list\_\_}.

\section{Hello, World!}\label{hello-world}

Pada Python 2.x, print (\texttt{std\ out}), menggunakan keyword
\texttt{print}, sedangkan pada Python 3.x, \texttt{print} dijadikan
sebagai fungsi \texttt{print(\textquotesingle{}hello,\ world}).

Untuk menampilkan \texttt{Hello,\ world!} di layar, buka REPL/IDLE
Python, kemudian ketikkan perintah:

\begin{verbatim}
print('Hello, world!')
\end{verbatim}

\section{Variable assignment}\label{variable-assignment}

Dalam membuat sebuah variabel, tidak diperlukan penulisan tipe data dari
variabel secara implisit.

misal:

\begin{verbatim}
phi = 3.14
\end{verbatim}

atau:

\begin{verbatim}
c = 3e+8
\end{verbatim}

\section{Tipe data primitif}\label{tipe-data-primitif}

\begin{itemize}
\tightlist
\item
  Integer
\end{itemize}

\begin{quote}
Bilangan bulat
\end{quote}

\begin{quote}
contoh: \texttt{num\ =\ 1337}
\end{quote}

\begin{itemize}
\tightlist
\item
  String
\end{itemize}

\begin{quote}
Karakter yang diapit oleh tanda petik dua
\texttt{\textquotesingle{}string\textquotesingle{}}
\end{quote}

\begin{quote}
contoh:
\end{quote}

\begin{quote}
\texttt{welcome\_message\ =\ "Hello\ \{\}!"\ \ print(welcome\_message.format(\textquotesingle{}Guido\textquotesingle{}))}
\end{quote}

\begin{itemize}
\item
  Float
\item
  List
\item
  Boolean
\item
  Dict
\item
  Tupple
\end{itemize}
